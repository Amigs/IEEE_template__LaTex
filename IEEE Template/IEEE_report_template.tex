\documentclass[journal, spanish]{IEEEtran}

% *** CITATION PACKAGES ***
\usepackage[style=ieee]{biblatex} 
\bibliography{example-bib.bib} %your file created using JabRef
\usepackage{hyperref}
\usepackage{amssymb}
% TIENG VIET THI UNCOMMENT DONG BEN DUOI
\usepackage[spanish]{babel}
\usepackage[utf8]{inputenc}
\usepackage{amsmath}
% *** MATH PACKAGES ***
\usepackage{amsmath}
\usepackage{multirow}
% *** PDF, URL AND HYPERLINK PACKAGES ***
\usepackage{url}
\usepackage{hyperref}
\hypersetup{
    colorlinks=true,
    linkcolor=blue,
    filecolor=magenta,      
    urlcolor=red,
}
% correct bad hyphenation here
\hyphenation{op-tical net-works semi-conduc-tor}
\usepackage{graphicx}  %needed to include png, eps figures
\usepackage{float}  % used to fix location of images i.e.\begin{figure}[H]

\begin{document}

% paper title
\title{Plantilla IEEE para plataformas \LaTeX \\
\small{Actividad de aprendizaje: Diseñar una plantilla con extensión .tex}}

% author names 
\author{Author1 name - correo@correo.edu.co \\Author2 name - correo1@correo.edu.co \\Author3 name - correo3@correo.edu.co \\Author4 name - correo4@correo.edu.co}
% <-this % stops a space
        
% The report headers
\markboth{Bogotá Colombia - \today}%do not delete next lines
{Shell \MakeLowercase{\textit{et al.}}: Bare Demo of IEEEtran.cls for IEEE Journals}

% make the title area
\maketitle

% As a general rule, do not put math, special symbols or citations
% in the abstract or keywords.
\begin{abstract} 
En el siguiente documento se pretende crear una plantilla IEEE LaTeX, en el cuál, también seŕa una guía de introduccioń a latex a dobble columna.
\end{abstract} 

\begin{IEEEkeywords}
iEEE,LaTeX,bib.\\
\end{IEEEkeywords}

\textbf{Reflexión:} Es indispensable al momento de presentar un informe de laboratorio, sustentación de proyecto de aula o tésis de grado ,y así, con todos los documentos que requieran un control absoluto sobre: el orden, la notación científica, el anexo de imágenes tablas, referenciar autores y la inserción de ecuaciones que, en ciertos sofware se realiza de forma aleatoria y/o no posee los carácteres especializados para plasmar: Una ecuación matemática o física, código extraido de una consola de programación y fórmulas químicas. Recuerde que  \textit{\textbf{“Somos lo que hacemos reiteradamente. La excelencia, por tanto, no es un acto, sino un hábito” \textsc{- Arístóteles.}}} \\¡¡Bienvenido a \LaTeX!!


\section{Introducción}
%\IEEEPARstart{D}{escribe:} 
Para este documento se tiene como propósito profundizar los conceptos y conocimientos en el manejo de LaTex y el editor OverLeaf, por lo que se planteará  una solución a un sistema de automatización para la fábrica de papas fritas, trabajadas desde el inicio del programa.
Primero analizaremos la planta comoi un sistema, es decir entradas y salidas.
\subsection{\bf Software utilizado}
\begin{itemize}
    \item LATeX.
    \item OpenSource.
\end{itemize}
\subsection{\bf GLOSARIO}
\begin{itemize}
    \item {\bf LaTeX:}which is pronounced «Lah-tech» or «Lay-tech» (to rhyme with «blech» or «Bertolt Brecht»), is a document preparation system for high-quality typesetting. It is most often used for medium-to-large technical or scientific documents but it can be used for almost any form of publishing.
\end{itemize}
\section{Análisis}

Se debe tener en cuenta que \LaTeX es un tipo de lenguaje muy similar a XML, es decir cada "item" nuevo que se abra, cada instrucción nueva debe tener su final, su cierre de ciclo. También se a agregado una carpeta llamada Images el cual contendra las imagenes utilizadas en el archivo, también se a agregado un archivo llamado example-bib.bib el cual es donde estarán todas las referencias, citas, para agregar a la bibliografía, el contenido de este archivo es de estructura .bib, para más información sobre .bib visita: \url{https://www.latex-tutorial.com/tutorials/bibtex/}. En nuestro caso utilizaremos \textit{Zotero} para agregar referencias fácilmente al informe.

\section{Procedimiento}
Con el fin de hacer de esta plantilla una guía para iniciar con Latex en el ámbito IEEE, en esta sección procederemos a incluir imagenes, tablas, lineas de código, citas, y así poder darle un aventón al mundo de las posiblidades que puedes hacer con \textbf{\LaTeX}.
\subsection{IMAGES}
\noindent
En la figura \ref{fig:led_Frit} podemos ver que está centrada y con un tamaño predefinido:
\begin{figure}[H]%[!ht]
    \begin {center}
    \includegraphics[width=6cm, height=4.5cm]{Images/Latex.jpeg}
    \caption{Manejo de imágenes en LaTeX.}
    \label{fig:led_Frit}
    \end {center}
\end{figure}

\noindent
También podemos colocar la imagen con inclinación como se puede apreciar en la figura \ref{fig:images}:
\begin{figure}[H]%[!ht]
    \centering
    \caption{Inclinando imágenes en LaTeX.}
    \includegraphics[width=6cm, height=4.5cm, angle=180]{Images/Latex.jpeg}
    \label{fig:images}
\end{figure}
\subsection{TABLES}
En está sección se abordará de manerá rápida como incluir tablas y trabajar con ellas.
    
\noindent
Para finalizar el laboratorio, se procede a utilizar el software OpenPLC para convertir el diagrama de GRAFCET de la figura 1, a lenguaje ladder (contactos).
En primer lugar se definen las variables en openPLC, las únicas son variable que hace referencia al sensor, y la motobomba que es el actuador de la planta.

Si el sensor tiene el valor de 0, es decir hay ausencia de agua, como esta negada toma el valor de 1 activando así la motobomba (actuador), en cambio de esta manera, si el sensor detecta presencia de agua, lo convierte a 0 y desactiva la motobomba.
\section{Conclusiones}
\begin{itemize}
    \item En la actualidad existen diversas herramientas digitales y Software que permiten profundizar en el aprendizaje remoto, además profundizar conceptos vistos en nuestras popias carreras.
    \item Los gŕaficos de GRAFCET son de bastante utilidad en el momento de automatizar, debido a que el proceso, transiciones y acciones se pueden ver fácilmente reflejados en él.
    \item la Domótica e inmótica son tecnologías que traen muchos beneficios a la hora de implemetar, siendo muy vieble estudiar estás tencnologías en relación a la automatización de procesos.

\end{itemize}

\end{document}


